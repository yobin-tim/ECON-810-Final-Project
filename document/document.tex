%\documentclass[12pt]{article}
 %% LyX 2.3.2-2 created this file.  For more info, see http://www.lyx.org/.
%% Do not edit unless you really know what you are doing.
\documentclass[12pt,english]{article}
\usepackage[T1]{fontenc}
\usepackage[latin9]{luainputenc}
\usepackage{geometry}
\geometry{verbose,tmargin=0.8in,bmargin=0.8in,lmargin=0.8in,rmargin=0.8in}
\usepackage{color}
\usepackage{babel}
\usepackage{float}
\usepackage{amsmath}
\usepackage{amsthm}
\usepackage{amssymb}
\usepackage{graphicx}
\usepackage{setspace}
\usepackage[authoryear]{natbib}
\onehalfspacing
\usepackage[unicode=true,pdfusetitle,
 bookmarks=true,bookmarksnumbered=false,bookmarksopen=false,
 breaklinks=false,pdfborder={0 0 1},backref=false,colorlinks=true]
 {hyperref}
\hypersetup{
 pdfborderstyle=,pdfborderstyle={},pdfborderstyle={},linkcolor=blue,urlcolor=blue,citecolor=blue,pdfstartview={FitH},hyperfootnotes=false}

\makeatletter

% %%%%%%%%%%%%%%%%%%%%%%%%%%%%%% LyX specific LaTeX commands.
% %% Because html converters don't know tabularnewline
% \providecommand{\tabularnewline}{\\}

%%%%%%%%%%%%%%%%%%%%%%%%%%%%%% User specified LaTeX commands.
%\documentclass[11pt]{article}
\usepackage{etex}
%\usepackage[round,longnamesfirst]{natbib}
\usepackage{amsfonts}
\usepackage{mathpazo}
\usepackage{hyperref}
\usepackage{xcolor}
\hypersetup{
	colorlinks,
	linkcolor={blue!75!black},
	citecolor={blue!75!black},
}
\usepackage{multimedia}
\usepackage{graphicx, color}
%\usepackage{pstricks,pst-node,fancybox,pst-text}
\usepackage{epsfig}
\usepackage{amsthm}
\usepackage{mathtools}
\usepackage{amsfonts}
\usepackage{fancyheadings}
\usepackage{float}
\usepackage{color}
\usepackage{mathrsfs}
\usepackage{setspace}
%\usepackage{lipsum}
%\usepackage{tikz}
\usepackage[mathscr]{euscript}
\usepackage{caption}
%\usepackage{subcaption}
\usepackage{pdflscape}
\usepackage{rotating}
\usepackage{booktabs}
%\usepackage[utf8]{inputenc}
\usepackage[T1]{fontenc}
\usepackage{geometry}
\newtheorem{thm}{Theorem}[section]
\newtheorem{cor}[thm]{Corollary}
\newtheorem{lem}[thm]{Lemma}
\newtheorem{prop}[thm]{Proposition}
\usepackage[bottom]{footmisc}

\newenvironment{definition}[1][Definition]{\begin{trivlist}
		\item[\hskip \labelsep {\bfseries #1}]}{\end{trivlist}}

\setlength{\topmargin}{-0.4in}
\setlength{\textheight}{8.85in}

%%  Edited to make stat review comments easier
\newif\ifStatReview% \StatReviewfalse
%%\StatReviewtrue    %   Show stat review comments
\StatReviewfalse    %   Ignore stat review comments
\ifStatReview
    \setlength{\oddsidemargin}{-0.2in}
    \setlength{\evensidemargin}{0.0in}
    \setlength{\textwidth}{6.0in}
    
    
    \usepackage{tikz}
    \let\oldmarginpar\marginpar
    % renew the \marginpar command to draw 
    % a node; it has a default setting which 
    % can be overwritten
    \renewcommand{\marginpar}[2][rectangle,draw,fill=yellow,rounded corners,text width=3.5cm]{%
            \oldmarginpar{%
            \tikz \node at (0,0) [#1]{#2};}%
            }
    \newcounter{statreview}
    \newenvironment{statreview}[1][]{\refstepcounter{statreview}\par\medskip
       \textbf{Stat Review~\thestatreview #1} \rmfamily}{\medskip}
  \else
    \setlength{\oddsidemargin}{-0.2in}
    \setlength{\evensidemargin}{0.0in}
    \setlength{\textwidth}{6.93in}
    \renewcommand{\marginpar}[2][]{}
\fi

\renewcommand{\baselinestretch}{1.44}

\usepackage{setspace}
\onehalfspacing

\@ifundefined{showcaptionsetup}{}{%
 \PassOptionsToPackage{caption=false}{subfig}}
\usepackage{subfig}
\makeatother
\usepackage{amsmath}
\usepackage{amsthm}
\usepackage{amssymb}

\begin{document}
\title{Final Project Setup}
\author{Yobin Timilsena \& Mitchell Valdes-Bobes}
\maketitle
% \begin{abstract}
% 	We examine [X]
% \end{abstract}
% \thispagestyle{empty}

% \pagebreak{}
% This paper .... 


\section{Model}
\begin{itemize}
	\item Households live for $ T $ periods, without retirement. Agents are heterogeneous in human capital $ h $ and assets $ k $. Agents are either employed or unemployed (but looking for a job). They spend $ s $ proportion of time in school each period.  Moreover, $ S $ represents cumulative years of schooling.
	There are two kinds of firms: a high type and a low type.
\end{itemize}

\subsection{Firms}
\begin{itemize}
    \item There are $2$ firm types $I \in \{L,H\}$, with $ \mu $ fraction of firms being the low type.
    \item Firm type $I=L$ hires all workers while firm type $I=H$ hires only workers with $S \geq \underline{S}$.
\end{itemize}

\subsection{Workers}
\begin{itemize}
    \item Law of motion of human capital is: $h' = \exp(z) H(h,s)$ where $z \in \mathbb{R}_+$ is a random shock. 
\end{itemize}
\subsubsection{Unemployed Workers}
\begin{itemize}
    \item Search for a job with intensity $\gamma$, $(\gamma + s = 1)$.
    \item Receive a job offer with probability $\pi(\gamma, S)$, $(\pi(0,\cdot) = 0)$ -- $ \pi_t(\gamma, S) = \gamma \cdot \frac{S}{t} $.
    \item Dependent on $S$ they might receive an offer from just $L$ or both firms.
    \item Receive unemployment benefits $b$ while unemployed.
    
    \item There are $3$ state variables:
    \begin{itemize}
    	\item $h \in \mathbb{R}_+$ human capital. Law of motion $ h' = \exp{z'} H(h,s) $.
    	\item $k \in \mathbb{R}_+$ assets.
    	\item $S \in \mathbb{R}_+$ (accumulated) schooling. Law of motion $ S' = S + s $.
    \end{itemize}
\end{itemize}

\textbf{Value Function if $S < \underline{S}$}

\begin{align*}
    U_t(h,k,S) =  \max_{k',s} \left\{ u(c) + \beta\mathbb{E}\left[\pi(\gamma, S) \cdot \mu \cdot W^L_{t+1}(h',k',S')
    + (1 - \pi(\gamma, S) \cdot \mu)U^L_{t+1}(h',k',S')  \right] \right\}
\end{align*}


\textbf{Value Function if $S \geq \underline{S}$}

\begin{align*}
	U_t(h,k,S) = &  \max_{k',s} \biggl\{ u(c) + \beta\mathbb{E}\biggl[\pi(\gamma, S) \left[\mu W^L_{t+1}(h',k',S') + (1 - \mu) W^H_{t+1}(h',k',S') \right]  \\
		+ & (1 - \pi(\gamma, S)) U^L_{t+1}(h',k',S')  \biggr] \biggr\}
\end{align*}
with the budget constraint \[  c + k' \leq b + k(1+r). \]

\subsubsection{Employed Workers}
\begin{itemize}
	\item Divide their time for $ s + l = 1 $.
	\item No on-the-job search allowed.
\end{itemize}

\textbf{Value function}
\begin{align*}
	W^I_t(h,k,S) =  \max_{k',s} \left\{ u(c) + \beta\mathbb{E}\left[ (1 - \delta) W^I_{t+1}(h',k',S')
	+ \delta  U_{t+1}(h',k',S')  \right] \right\}
\end{align*}
with the budget constraint \[   c + k' \leq R^I_t h l + k(1 +r) \]
% \begin{align*}
%     U_t(h,k,S) = \\
%     &\max_{k,s} \left\{ u(c) + \beta\mathbb{E}\left[\pi(\gamma, S)\left(\mu W^L_{t+1}(h',k',S')
%     + (1 - \pi(\gamma, S))U^L_{t+1}(h',k',S')  \right] \right\}
% \end{align*}




%\bibliographystyle{chicago}
%\bibliography{references}


\end{document}